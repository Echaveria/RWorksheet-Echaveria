% Options for packages loaded elsewhere
\PassOptionsToPackage{unicode}{hyperref}
\PassOptionsToPackage{hyphens}{url}
%
\documentclass[
]{article}
\usepackage{amsmath,amssymb}
\usepackage{iftex}
\ifPDFTeX
  \usepackage[T1]{fontenc}
  \usepackage[utf8]{inputenc}
  \usepackage{textcomp} % provide euro and other symbols
\else % if luatex or xetex
  \usepackage{unicode-math} % this also loads fontspec
  \defaultfontfeatures{Scale=MatchLowercase}
  \defaultfontfeatures[\rmfamily]{Ligatures=TeX,Scale=1}
\fi
\usepackage{lmodern}
\ifPDFTeX\else
  % xetex/luatex font selection
\fi
% Use upquote if available, for straight quotes in verbatim environments
\IfFileExists{upquote.sty}{\usepackage{upquote}}{}
\IfFileExists{microtype.sty}{% use microtype if available
  \usepackage[]{microtype}
  \UseMicrotypeSet[protrusion]{basicmath} % disable protrusion for tt fonts
}{}
\makeatletter
\@ifundefined{KOMAClassName}{% if non-KOMA class
  \IfFileExists{parskip.sty}{%
    \usepackage{parskip}
  }{% else
    \setlength{\parindent}{0pt}
    \setlength{\parskip}{6pt plus 2pt minus 1pt}}
}{% if KOMA class
  \KOMAoptions{parskip=half}}
\makeatother
\usepackage{xcolor}
\usepackage[margin=1in]{geometry}
\usepackage{graphicx}
\makeatletter
\def\maxwidth{\ifdim\Gin@nat@width>\linewidth\linewidth\else\Gin@nat@width\fi}
\def\maxheight{\ifdim\Gin@nat@height>\textheight\textheight\else\Gin@nat@height\fi}
\makeatother
% Scale images if necessary, so that they will not overflow the page
% margins by default, and it is still possible to overwrite the defaults
% using explicit options in \includegraphics[width, height, ...]{}
\setkeys{Gin}{width=\maxwidth,height=\maxheight,keepaspectratio}
% Set default figure placement to htbp
\makeatletter
\def\fps@figure{htbp}
\makeatother
\setlength{\emergencystretch}{3em} % prevent overfull lines
\providecommand{\tightlist}{%
  \setlength{\itemsep}{0pt}\setlength{\parskip}{0pt}}
\setcounter{secnumdepth}{-\maxdimen} % remove section numbering
\ifLuaTeX
  \usepackage{selnolig}  % disable illegal ligatures
\fi
\IfFileExists{bookmark.sty}{\usepackage{bookmark}}{\usepackage{hyperref}}
\IfFileExists{xurl.sty}{\usepackage{xurl}}{} % add URL line breaks if available
\urlstyle{same}
\hypersetup{
  pdftitle={RWorksheet\_Echaveria\#4b},
  pdfauthor={Luigi Echaveria},
  hidelinks,
  pdfcreator={LaTeX via pandoc}}

\title{RWorksheet\_Echaveria\#4b}
\author{Luigi Echaveria}
\date{2023-11-08}

\begin{document}
\maketitle

\#1. Using the for loop, create an R script that will display a 5x5
matrix as shown in Figure 1. It must contain vectorA = {[}1,2,3,4,5{]}
and a 5 x 5 zero matrix.

vector A \textless-c(1, 2, 3, 4, 5)

matrixA \textless- matrix(0, nrow = 5, ncol = 5)

for (i in 1:5) \{ for (j in 1:5) \{ matrixA{[}i, j{]} \textless-
abs(vectorA{[}i{]} - vectorA{[}j{]}) \} \}

print(matrixA)

\#2 Print the string ``*'' using for() function. The output should be
the same as shown in Figure

for (c in 1:5)\{ cat(paste0(``"'',rep(``*``,c)''"``),''\n``\,``) \}

\#3 Get an input from the user to print the Fibonacci sequence starting
from the 1st input up to 500. Use repeat and break statements. Write the
R Scripts and its output.

n \textless- as.integer(readline(``Enter a number to start the Fibonacci
sequence:''))

fib \textless- c(0, 1)

cat(``Fibonacci Sequence:\n'') cat(fib{[}2{]}, '' ``)

repeat \{ next\_fib \textless- fib{[}length(fib){]} + fib{[}length(fib)
- 1{]} if (next\_fib \textgreater{} 500) \{ break \} fib \textless-
c(fib, next\_fib) cat(next\_fib, '' ``) \}

cat(``\n'')

\end{document}
